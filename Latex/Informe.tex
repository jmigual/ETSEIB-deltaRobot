\documentclass[a4paper, 12pt]{article}

\usepackage[dvipsnames]{xcolor} % Code highlighting color
\usepackage[catalan]{babel} % Language 
\usepackage{fontspec} 
\usepackage{fullpage}
\usepackage[a4paper, margin=2cm]{geometry} % To change the margins
\usepackage{graphicx} % Insert images
\usepackage[hidelinks]{hyperref} % Links color
\usepackage{import}
\usepackage[final]{pdfpages}
\usepackage{ragged2e}
\usepackage{wrapfig} %To Text wrap
\usepackage{listings} % Add code
\usepackage{tikz}
\usepackage{verbatim}
\usepackage{menukeys}

\usetikzlibrary{shapes, arrows, positioning}
\tikzstyle{startstop} = [rectangle, rounded corners, minimum width=3cm, minimum height=1cm,text centered, draw, fill=red!30]
\tikzstyle{outcalc} = [rectangle, rounded corners, minimum width=3cm, minimum height=1cm,text centered, draw=black, fill=yellow!30]
\tikzstyle{io} = [trapezium, trapezium left angle=70, trapezium right angle=110, minimum width=3cm, minimum height=1cm, text centered, draw=black, fill=orange!40]
\tikzstyle{process} = [rectangle, minimum width=3cm, minimum height=1cm, text centered, draw=black, fill=blue!30]
\tikzstyle{decision} = [diamond, aspect=2, minimum width=3cm, minimum height=0.5cm, text centered, draw=black, fill=green!30]
\tikzstyle{arrow} = [thick,->,>=stealth]

\lstset{
language=[Sharp]C,
basicstyle=\ttfamily, 
keywordstyle=\color{cyan}, 
stringstyle=\color{orange}, 
commentstyle=\color{OliveGreen},
numbers=left,
morekeywords={partial, var, value, get, set},
basicstyle=\ttfamily\scriptsize,
breaklines=true
}

\renewcommand*\contentsname{Índex}
\setlength\parindent{0pt}

\begin{document}
\title{\textsc{Informe Projecte I} \\ \large Construcció d'un robot delta per co\lgem ocar un circuit de dominó}
\author{Marc Asenjo i Ponce de León \and
		Joan Marcè i Igual \and
		Iñigo Moreno i Caireta}
\date{\today}
\maketitle
\begin{center}
\includegraphics[width=0.4\textwidth]{./imgComp/logo}
\end{center}

\newpage
\tableofcontents{}

\newpage
\section{Objectiu}
El nostre objectiu és construir un robot delta que sigui capaç de posicionar peces de dominó de la manera desitjada per l'usuari, aquest tindrà un programa que li permetrà dibuixar el circuit desitjat i també podrà controlar el robot mitjançant un joystick.

\newpage
\input{./mecanic}

\newpage
\section{Informe cinemàtic}
\subsection{Cinemàtica inversa}

El càlcul invers és el que serveix per trobar els angles a partir de la posició a la que es vol situar la plataforma. Es pot fer el càlcul per cada motor de manera separada. Primer s'han de definir les mides principals del robot que s'usaran en el càlcul.

\subsubsection{Definicions de variables}
\begin{figure}[h!]
\centering
\includegraphics[width=12cm]{./imgComp/esquema_general}
\end{figure}

\begin{description}
\item[a] és la mida total del braç, des de l'eix del motor fins a l'eix de la connexió amb l'avantbraç.
\item[b] és la mida de tot l'avantbraç des de la connexió amb el braç fins a la unió amb la plataforma.
\item[L1] és la distància entre el centre de la base als eixos dels motors.
\item[L2] és la distància entre el centre de la plataforma a l'eix de connexió amb l'avantbraç.
\end{description}

\subsubsection{Canvis de base}
Com el càlcul de cada angle és independent de la resta, es poden fer canvis de base per poder fer-ho tot amb una sola funció a l'hora de programar. La base inicial $\{x_0,y_0,z_0\}$ està al centre de la base, amb la $x$ en la direcció del motor 1. Les bases $\{x_i,y_i,z_i\}$ que s'utilitzaran per calcular l'angle del motor $i$ estan posicionades al centre del eix de cada motor, i la seva $x$ apunta en la direcció perpendicular al eix del motor. També es suma $L2$ a $x_i$ per fer que la posició objectiu sigui el punt de connexió entre la plataforma i l'avantbraç en lloc del centre de la plataforma.

\begin{figure}[h!]
\centering
\includegraphics[width=7cm]{./sketch/canvi_base}
\end{figure}

$$\{x_1,y_1,z_1\}=\{x_0+L2-L1,y_0,z_0\}$$
$$\{x_2,y_2,z_2\}=\{z_0sin(60)-x_0cos(60)+L2-L1,y_0,-z_0cos(60)-x_0sin(60)\}$$
$$\{x_3,y_3,z_3\}=\{-z_0sin(60)-x_0cos(60)+L2-L1,y_0,-z_0cos(60)+x_0sin(60)\}$$

\subsubsection{Càlcul d'un angle}

El motor només pot moure el final del braç en un cercle, gràcies a això es poden deduir dues formules:$$x^2+y^2=a^2 \quad \textrm{i} \quad z=0$$

\begin{figure}[h!]
\centering
\includegraphics[width=8cm]{./sketch/calcul_angle}
\end{figure}

També es pot veure que, com ha d'estar connectat a la plataforma amb una distancia $b$ mitjançant l'avantbraç, s'ha de complir la formula: \[(x-x_i)^2+(y-y_i)^2+(z-z_i)^2=b^2\]
Ara només cal resoldre el sistema d'equacions.

\[x^2 - 2xx_i + x_i^2 + y^2 - 2yy_i + y_i^2 + z_i^2 = b^2\]
\[-2xx_i - 2yy_i = \underbrace{b^2 - a^2 - x_i^2 - y_i^2 - z_i^2}_{n}\]
\[-2x_i\sqrt{a^2-y^2}=n+2yy_i\]
\[4a^2x_i^2-4y^2x_i^2=n^2+4yy_in+4y^2y_i^2\]
\[y^2(x_i^2+y_i^2)+y(y_in)+(\frac{n^2}{4}-a^2x_i^2)=0\]
\[y=\frac{-y_i\pm\sqrt{y_i^2n^2-4(x_i^2+y_i^2)(\frac{n^2}{4}-a^2x_i^2)}}{2(x_i^2+y_i^2)}\]
\[x=\pm\sqrt{a^2-y^2}\]
\[\textrm{Finalment, }\theta=atan(\frac{y}{x})\]

\subsection{Comprovació dels càlculs}

Per poder provar les nostres funcions sense trencar el robot, es va decidir fer una simulació en del Robot en Unity. La simulació és molt simple, Unity s'encarrega de dibuixar un esquema 3D del robot utilitzant línies a partir de la posició desitjada de la plataforma i els angles del robot calculats amb la funció trobada.

\begin{figure}[h!]
\centering
\includegraphics[width=7cm]{./images/simulacioRobot}
\end{figure}

Aquesta aplicació també es va usar per determinar quin símbol s'ha de posar abans de les dos arrels. Per la primera arrel es va determinar que havia de anar acompanyada d'un signe negatiu només quan \(x_i\) és negatiu. La segona arrel només ha de ser negativa a partir de quan el ha d'estar completament cap a dalt. La condició exacta es: \(b^2-(y_i+a)^2<x_i^2+z_i^2\) 

\newpage
\subsection{Rang de treball}

Per calcular el rang de treball s'ha utilitzat la fórmula de l'apartat anterior i buscant una gran quantitat de punts, d'aquests els que donaven un resultat possible s'han agafat com a vàlids i els altres s'han descartat. Tot seguit s'han passat els punts a Matlab\textsuperscript{\textregistered} i s'ha generat la següent superfície.

\begin{figure}[h!]
\centering
\begin{minipage}{7cm}
\centering
\includegraphics[width=7cm]{./images/rangTreball}
\end{minipage}
\\
\hfill
\begin{minipage}{7cm}
\centering
\includegraphics[width=7cm]{./images/rangTreball2}
\end{minipage}
\hfill
\begin{minipage}{7cm}
\centering
\includegraphics[width=7cm]{./images/rangTreball3}
\end{minipage}
\hfill
\end{figure}

Aquesta superfície està calculada amb tots els punts possibles teòrics però el robot te moltes limitacions mecàniques que no es tenen en compte en aquest càlcul. Per exemple els paral·lelograms dels avantbraços no poden tenir un angle qualsevol, ja que les barres de metall xoquen contra les plaques de fusta del braç si l'angle es molt petit.

\clearpage
\subsection{Implementació en Matlab}

La implementació de Matlab té tres parts, dues per al càlcul dels angles de treball i la tercera per poder calcular el rang de treball del robot.

\subsubsection{Càlcul d'un angle}
\lstinputlisting[language=Matlab]{../Matlab/singleAngle.m}

\newpage
\subsubsection{Càlcul de tots els angles}
\lstinputlisting[language=Matlab]{../Matlab/setAngles.m}

\newpage
\subsubsection{Càlcul del rang de treball}
\lstinputlisting[language=Matlab]{../Matlab/calcWorkspace.m}


\newpage
\section{Informe de programació}

La programació esta dividida en dos programes, el de Control i el de Unity. El primer és el que fa "moure el robot" i el segon és el que permet dibuixar el circuit que posteriorment serà col·locat.
Tot el codi es pot trobar al següent repositori: \href{https://github.com/jmigual/deltaRobot}{https://github.com/jmigual/deltaRobot}.
Per descarregar els programes cal anar a l'apartat de \verb|releases| i allà s'ha d'escollir la versió que es vulgui descarregar.

\subsection{Control automàtic}
\subsubsection{Descripció del programa}

Per fer el control automàtic s'ha utilitzat Qt 5.4 i també la API en C de Dynamixel de comunicació amb els servos. En aquest informe no es detalla molt aquesta part ja que a l'annex hi ha la documentació completa.

Al programa se li carrega un arxiu que conté totes les peces de dominó (extensió \verb|.df|) que s'han de col·locar i executar el programa. Aquest llegeix l'arxiu i, mitjançant l'algoritme descrit a continuació, les va posant al seu lloc d'una en una. L'arxiu generalment l'haurà generat la part de Unity.

Per col·locar una peça es divideix la trajectòria en múltiples passos, aquí cada posició en un pas concret per a un dominó en concret es denota \verb|posicio[domino][pas]|.

\subsubsection{Us del programa}

Per fer anar el programa de control primer cal anar a \verb|Edit -> Options| i allà seleccionar els ID dels servos que s'utilitzen per a cada braç, el braç que està a sobre la plataforma és el 1 i els altres dos són el 2 i el 3 en sentit antihorari.
Un cop seleccionat els servos llavors es pot seleccionar el mode:
\begin{description}
\item[Manual] permet controlar el robot mitjançant les tecles \verb|W, A, S, D| per moure'l; \verb|Q, E| per pujar i baixar i \verb|H, J| per girar la punta del robot.
\item[Automàtic] començarà a executar el programa que se li hagi carregat prèviament, per carregar un programa anar a \verb|File -> Import|.
\end{description}


\begin{tikzpicture}[node distance=1cm, auto, align=center]
\node(start)[startstop] {Inici};
\node(init)[process, below = of start]{\verb|domino = 1|};
\node(take)[process, below = of init, text width = 4cm]{Anar a posició d'espera per sota};
\node(takeUp)[process, below = of take, text width = 4cm]{Augmentar altura fins altura de treball};

\node(waitRead)[io, below = of takeUp]{Llegir teclat};
\node(waitDeci)[decision, below = of waitRead]{Enter premut?};

\node(rotate)[process, below = of waitDeci, text width=5cm]{Girar peça els angles especificats};
\node(init2)[process, below = of rotate]{pas = 1};

\node(going)[node distance = 3.7, process, right = of start]{\verb|pos = posicio[domino][pas]|};
\node(going2)[process, below = of going, text width=4cm]{Ordenar als servos anar a l'angle de 'pos'};
\node(goingRead)[io, below = of going2, text width=3cm]{Llegir posició servos};
\node(goingIf)[decision, below = of goingRead]{\verb|pos == posicio servos ?|};
\node(goingEnd)[decision, below = of goingIf]{\verb|pas == passos[domino] ?|};
\node(goingInc)[process, right = of goingEnd, node distance = 0.8cm]{\verb|pas = pas + 1|};
\node(goingEnd2)[decision, below = of goingEnd]{\verb|domino == nº dominos ?|};
\node(goingEnd3)[process, below = of goingEnd2]{\verb|domino = domino + 1|};
\node(end)[startstop, right = of goingEnd2, node distance = 0.8cm]{Fi};


\node(precalc)[outcalc, above = of start, node distance = 0.5cm]{Precàlcul};

\draw [arrow] (start) -- (init);
\draw [arrow] (init) -- (take);
\draw [arrow] (take) -- (takeUp);
\draw [arrow] (takeUp) -- (waitRead);

\draw [arrow] (waitRead) -- (waitDeci);
\draw [arrow] (waitDeci) --++ (3, 0) node[near start]{no} |- (waitRead);
\draw [arrow] (waitDeci) -- node[near start] {si} (rotate);

\draw [arrow] (rotate) -- (init2);
\draw [arrow] (init2) --++ (3.5, 0) |- (going);

\draw [arrow] (going) -- (going2);
\draw [arrow] (going2) -- (goingRead);
\draw [arrow] (goingRead) -- (goingIf);
\draw [arrow] (goingIf) --++ (4, 0) node[near start]{no} |- (goingRead);
\draw [arrow] (goingIf) -- node[near start]{si} (goingEnd);
\draw [arrow] (goingEnd) -- node[near start]{no} (goingInc);
\draw [arrow] (goingInc) |- (going);
\draw [arrow] (goingEnd) -- node[near start]{si} (goingEnd2);
\draw [arrow] (goingEnd2) -- node[near start]{no} (goingEnd3);

\draw [arrow] (goingEnd3) --++(-4, 0) |- (take);
\draw [arrow] (goingEnd2) -- node[near start]{si} (end);
\end{tikzpicture}


\subsection{Unity}

En Unity s'han fet dos escenes, la principal (Main) i la del simulador del robot (\verb|SimuladorRobot|). Els scripts que s'han usat al Main són: 
\begin{itemize}
\item \verb|CameraController|: Permet moure la càmera i fer zoom.
\lstinputlisting[tabsize=2]{../Unity/Delta/Assets/Scripts/CameraController.cs}

\item \verb|dominoPosition|: Mou les peces de domino mentre les estas col·locant.
\lstinputlisting[tabsize=2]{../Unity/Delta/Assets/Scripts/dominoPosition.cs}

\item \verb|Draw|: Afegeix tota la funcionalitat de dibuixar circuits de domino.
\lstinputlisting[tabsize=2]{../Unity/Delta/Assets/Scripts/Draw.cs}

\item \verb|Editor|: Permet moure i rotar els objectes ja col·locats.
\lstinputlisting[tabsize=2]{../Unity/Delta/Assets/Scripts/Editor.cs}
\end{itemize}
Els scripts que s'han usat a \verb|SimuladorRobot| són:
\begin{itemize}
\item \verb|CameraController|: (el mateix que a Main).
\item \verb|RobotCreator|: Dibuixa el robot en 3D a partir dels paràmetres públics que se li passin.
\lstinputlisting[tabsize=2]{../Unity/Delta/Assets/Scripts/RobotCreator.cs}
\item \verb|Movements|: Modifica els paràmetres públics del robot per provar moviments determinats.
\lstinputlisting[tabsize=2]{../Unity/Delta/Assets/Scripts/Movements.cs}
\end{itemize}






\end{document}