\section{Informe de programació}

La programació esta dividida en dos programes, el de Control i el de Unity. El primer és el que fa "moure el robot" i el segon és el que permet dibuixar el circuit que posteriorment serà col·locat.
Tot el codi es pot trobar al següent repositori: \href{https://github.com/jmigual/deltaRobot}{https://github.com/jmigual/deltaRobot}.
Per descarregar els programes cal anar a l'apartat de \verb|releases| i allà s'ha d'escollir la versió que es vulgui descarregar.

\subsection{Control automàtic}
\subsubsection{Descripció del programa}

Per fer el control automàtic s'ha utilitzat Qt 5.4 i també la API en C de Dynamixel de comunicació amb els servos. En aquest informe no es detalla molt aquesta part ja que a l'annex hi ha la documentació completa.

Al programa se li carrega un arxiu que conté totes les peces de dominó (extensió \verb|.df|) que s'han de col·locar i executar el programa. Aquest llegeix l'arxiu i, mitjançant l'algoritme descrit a continuació, les va posant al seu lloc d'una en una. L'arxiu generalment l'haurà generat la part de Unity.

Per col·locar una peça es divideix la trajectòria en múltiples passos, aquí cada posició en un pas concret per a un dominó en concret es denota \verb|posicio[domino][pas]|.

\subsubsection{Ús del programa}

Per fer anar el programa de control primer cal anar a \menu[,]{Edit, Options} i allà seleccionar els ID dels servos que s'utilitzen per a cada braç, el braç que està a sobre la plataforma és el 1 i els altres dos són el 2 i el 3 en sentit antihorari.
Un cop seleccionat els servos llavors es pot seleccionar el mode:
\begin{description}
\item[Manual] permet controlar el robot mitjançant les tecles \keys{W}, \keys{A}, \keys{S}, \keys{D} per moure'l; \keys{Q}, \keys{E} per pujar i baixar i \keys{H}, \keys{J} per girar la punta del robot.
\item[Automàtic] començarà a executar el programa que se li hagi carregat prèviament, per carregar un programa anar a \menu[,]{File, Import}. Cada vegada que se li posa una peça a la posició marcada s'ha de prémer \keys{\return}.
\end{description}


\begin{tikzpicture}[node distance=1cm, auto, align=center]
\node(start)[startstop] {Inici};
\node(init)[process, below = of start]{\verb|domino = 1|};
\node(take)[process, below = of init, text width = 4cm]{Anar a posició d'espera per sota};
\node(takeUp)[process, below = of take, text width = 4cm]{Augmentar altura fins altura de treball};

\node(waitRead)[io, below = of takeUp]{Llegir teclat};
\node(waitDeci)[decision, below = of waitRead]{\keys{\return} premut?};

\node(rotate)[process, below = of waitDeci, text width=5cm]{Girar peça els angles especificats};
\node(init2)[process, below = of rotate]{pas = 1};

\node(going)[node distance = 3.7, process, right = of start]{\verb|pos = posicio[domino][pas]|};
\node(going2)[process, below = of going, text width=4cm]{Ordenar als servos anar a l'angle de 'pos'};
\node(goingRead)[io, below = of going2, text width=3cm]{Llegir posició servos};
\node(goingIf)[decision, below = of goingRead]{\verb|pos == posicio servos ?|};
\node(goingEnd)[decision, below = of goingIf]{\verb|pas == passos[domino] ?|};
\node(goingInc)[process, right = of goingEnd, node distance = 0.8cm]{\verb|pas = pas + 1|};
\node(goingEnd2)[decision, below = of goingEnd]{\verb|domino == nº dominos ?|};
\node(goingEnd3)[process, below = of goingEnd2]{\verb|domino = domino + 1|};
\node(end)[startstop, right = of goingEnd2, node distance = 0.8cm]{Fi};


\node(precalc)[outcalc, above = of start, node distance = 0.5cm]{Precàlcul};

\draw [arrow] (start) -- (init);
\draw [arrow] (init) -- (take);
\draw [arrow] (take) -- (takeUp);
\draw [arrow] (takeUp) -- (waitRead);

\draw [arrow] (waitRead) -- (waitDeci);
\draw [arrow] (waitDeci) --++ (3, 0) node[near start]{no} |- (waitRead);
\draw [arrow] (waitDeci) -- node[near start] {si} (rotate);

\draw [arrow] (rotate) -- (init2);
\draw [arrow] (init2) --++ (3.5, 0) |- (going);

\draw [arrow] (going) -- (going2);
\draw [arrow] (going2) -- (goingRead);
\draw [arrow] (goingRead) -- (goingIf);
\draw [arrow] (goingIf) --++ (4, 0) node[near start]{no} |- (goingRead);
\draw [arrow] (goingIf) -- node[near start]{si} (goingEnd);
\draw [arrow] (goingEnd) -- node[near start]{no} (goingInc);
\draw [arrow] (goingInc) |- (going);
\draw [arrow] (goingEnd) -- node[near start]{si} (goingEnd2);
\draw [arrow] (goingEnd2) -- node[near start]{no} (goingEnd3);

\draw [arrow] (goingEnd3) --++(-4, 0) |- (take);
\draw [arrow] (goingEnd2) -- node[near start]{si} (end);
\end{tikzpicture}


\subsection{Unity}

En Unity s'han fet dos escenes, la principal (Main) i la del simulador del robot (\verb|SimuladorRobot|). Els scripts que s'han usat al Main són: 
\begin{itemize}
\item \verb|CameraController|: Permet moure la càmera i fer zoom.
\lstinputlisting[tabsize=2]{../Unity/Delta/Assets/Scripts/CameraController.cs}

\item \verb|dominoPosition|: Mou les peces de domino mentre les estas col·locant.
\lstinputlisting[tabsize=2]{../Unity/Delta/Assets/Scripts/dominoPosition.cs}

\item \verb|Draw|: Afegeix tota la funcionalitat de dibuixar circuits de domino.
\lstinputlisting[tabsize=2]{../Unity/Delta/Assets/Scripts/Draw.cs}

\item \verb|Editor|: Permet moure i rotar els objectes ja col·locats.
\lstinputlisting[tabsize=2]{../Unity/Delta/Assets/Scripts/Editor.cs}
\end{itemize}
Els scripts que s'han usat a \verb|SimuladorRobot| són:
\begin{itemize}
\item \verb|CameraController|: (el mateix que a Main).
\item \verb|RobotCreator|: Dibuixa el robot en 3D a partir dels paràmetres públics que se li passin.
\lstinputlisting[tabsize=2]{../Unity/Delta/Assets/Scripts/RobotCreator.cs}
\item \verb|Movements|: Modifica els paràmetres públics del robot per provar moviments determinats.
\lstinputlisting[tabsize=2]{../Unity/Delta/Assets/Scripts/Movements.cs}
\end{itemize}




