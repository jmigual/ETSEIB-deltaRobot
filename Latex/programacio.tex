\section{Informe de programació}

La programació esta dividida en dos programes, el de Control i el de Unity. El primer és el que fa "moure el robot" i el segon és el que permet dibuixar el circuit que posteriorment serà col·locat.
Tot el codi es pot trobar al següent repositori: \href{https://github.com/jmigual/deltaRobot}{https://github.com/jmigual/deltaRobot}.
Per descarregar els programes cal anar a l'apartat de \verb|releases| i allà s'ha d'escollir la versió que es vulgui descarregar.

\subsection{Control automàtic}
\subsubsection{Descripció del programa}

Per fer el control automàtic s'ha utilitzat Qt 5.4 i també la API en C de Dynamixel de comunicació amb els servos. En aquest informe no es detalla molt aquesta part ja que a l'annex hi ha la documentació completa.

Al programa se li carrega un arxiu que conté totes les peces de dominó (extensió \verb|.df|) que s'han de col·locar i executar el programa. Aquest llegeix l'arxiu i, mitjançant l'algoritme descrit a continuació, les va posant al seu lloc d'una en una. L'arxiu generalment l'haurà generat la part de Unity.

Per col·locar una peça es divideix la trajectòria en múltiples passos, aquí cada posició en un pas concret per a un dominó en concret es denota \verb|posicio[domino][pas]|.

\subsubsection{Ús del programa}

Per fer anar el programa de control primer cal anar a \menu[,]{Edit, Options} i allà seleccionar els ID dels servos que s'utilitzen per a cada braç, el braç que està a sobre la plataforma és el 1 i els altres dos són el 2 i el 3 en sentit antihorari.
Un cop seleccionat els servos llavors es pot seleccionar el mode:
\begin{description}
\item[Manual] permet controlar el robot mitjançant les tecles \keys{W}, \keys{A}, \keys{S}, \keys{D} per moure'l; \keys{Q}, \keys{E} per pujar i baixar i \keys{H}, \keys{J} per girar la punta del robot.
\item[Automàtic] començarà a executar el programa que se li hagi carregat prèviament, per carregar un programa anar a \menu[,]{File, Import}. Cada vegada que se li posa una peça a la posició marcada s'ha de prémer \keys{\return}.
\end{description}


\begin{tikzpicture}[node distance=1cm, auto, align=center]
\node(start)[startstop] {Inici};
\node(init)[process, below = of start]{\verb|domino = 1|};
\node(take)[process, below = of init, text width = 4cm]{Anar a posició d'espera per sota};
\node(takeUp)[process, below = of take, text width = 4cm]{Augmentar altura fins altura de treball};

\node(waitRead)[io, below = of takeUp]{Llegir teclat};
\node(waitDeci)[decision, below = of waitRead]{\keys{\return} premut?};

\node(rotate)[process, below = of waitDeci, text width=5cm]{Girar peça els angles especificats};
\node(init2)[process, below = of rotate]{pas = 1};

\node(going)[node distance = 3.7, process, right = of start]{\verb|pos = posicio[domino][pas]|};
\node(going2)[process, below = of going, text width=4cm]{Ordenar als servos anar a l'angle de 'pos'};
\node(goingRead)[io, below = of going2, text width=3cm]{Llegir posició servos};
\node(goingIf)[decision, below = of goingRead]{\verb|pos == posicio servos ?|};
\node(goingEnd)[decision, below = of goingIf]{\verb|pas == passos[domino] ?|};
\node(goingInc)[process, right = of goingEnd, node distance = 0.8cm]{\verb|pas = pas + 1|};
\node(goingEnd2)[decision, below = of goingEnd]{\verb|domino == nº dominos ?|};
\node(goingEnd3)[process, below = of goingEnd2]{\verb|domino = domino + 1|};
\node(end)[startstop, right = of goingEnd2, node distance = 0.8cm]{Fi};


\node(precalc)[outcalc, above = of start, node distance = 0.5cm]{Precàlcul};

\draw [arrow] (start) -- (init);
\draw [arrow] (init) -- (take);
\draw [arrow] (take) -- (takeUp);
\draw [arrow] (takeUp) -- (waitRead);

\draw [arrow] (waitRead) -- (waitDeci);
\draw [arrow] (waitDeci) --++ (3, 0) node[near start]{no} |- (waitRead);
\draw [arrow] (waitDeci) -- node[near start] {si} (rotate);

\draw [arrow] (rotate) -- (init2);
\draw [arrow] (init2) --++ (3.5, 0) |- (going);

\draw [arrow] (going) -- (going2);
\draw [arrow] (going2) -- (goingRead);
\draw [arrow] (goingRead) -- (goingIf);
\draw [arrow] (goingIf) --++ (4, 0) node[near start]{no} |- (goingRead);
\draw [arrow] (goingIf) -- node[near start]{si} (goingEnd);
\draw [arrow] (goingEnd) -- node[near start]{no} (goingInc);
\draw [arrow] (goingInc) |- (going);
\draw [arrow] (goingEnd) -- node[near start]{si} (goingEnd2);
\draw [arrow] (goingEnd2) -- node[near start]{no} (goingEnd3);

\draw [arrow] (goingEnd3) --++(-4, 0) |- (take);
\draw [arrow] (goingEnd2) -- node[near start]{si} (end);
\end{tikzpicture}

\newpage
\subsection{Unity}

\subsubsection{El programa}

En Unity s'ha volgut fer un editor complet en el que es pugui navegar per l'espai de treball, afegir peces amb diferents eines, provar si el circuit dissenyat és correcte i finalment enviar la posició i angle de gir de cada peça al programa que controla el robot. Funciona de la següent manera:

Un cop s'executa el programa es troba en el mode editor on el temps està aturat. En aquest mode es pot moure la càmera utilitzant els botons \keys{W} \keys{A} \keys{S} \keys{D} i canviar l'orientació fent clic i arrossegant el botó dret del ratolí. Si es fa clic central en un objecte la càmera automàticament es centrarà en ell.

En quant a editar, hi ha dues maneres de fer-ho. Es pot clicar el botó de baix a l'esquerra amb la imatge d'una peça de domino i automàticament apareixerà una peça al cursor que es pot moure lliurement pel pla del terra, i deixar-la tornant a clicar (botó esquerre del ratolí). Un cop fet això es pot seleccionar la peça, que es posarà de color vermell. Si es prem \keys{T} apareixeran uns eixos que serveixen per acabar de col·locar de forma precisa la peça clicant i arrossegant l'eix que desitgem amb el clic esquerre. El mateix es pot fer per rotar la peça prement \keys{R}, encara que en aquest cas només és útil una rotació, la de l'eix vertical. En qualsevol moment es pot cancel·lar el mode editor actual tornant a clicar la tecla amb què s'ha iniciat (\keys{R} o \keys{T}). Seguint exactament el mateix procediment però clicant el botó de \keys{Add Force} es poden afegir unes fletxes que serviran de forces per comprovar que el circuit funciona.

La segona manera i la més còmode d'utilitzar per fer circuits de peces és la de "dibuixar". Si es clica \keys{Let's draw!} la càmera automàticament es posarà en una vista aèria. Si ara es fa clic esquerre comença a posar peces per on es fa passar el ratolí, fins que es torna a fer clic. Les peces automàticament es posen en la direcció i distància les unes de les altres necessàries perquè el circuit funcioni, però pot ser que hagi sortit malament, així que si es prem \keys{\del} estant en el mode de dibuix totes les peces s'eliminen. Per acabar, es prem \keys{\esc} i torna al mode editor, on es poden retocar les peces que es vulguin.

Per acabar, es fa clic a \keys{Play}. La fletxa avançarà endavant fins trobar-se una peça de dominó, empenyent-la i començant la reacció en cadena del circuit. Es pot utilitzar la barra de dalt a la dreta per augmentar o disminuir la velocitat en la que cauen les peces, encara que no és recomanable pujar-la massa per qüestions de rendiment del programa. Si s'acaba satisfet amb el comportament del circuit, es pot tornar a clicar el botó que abans es deia \keys{Play}, que ara té el nom de \keys{Pause}. Les peces es tornaran a posar com estaven abans de ferles caure i llavors es pot acabar amb el botó \keys{Save State}. Aquest botó actualitzara el document que s'utilitza per informar al programa principal de la posició de les peces.


\subsubsection{Documentació del codi}
En Unity s'han fet dos escenes, la principal (Main) i la del simulador del robot (\verb|SimuladorRobot|). Els scripts que s'han usat al Main són: 
\begin{itemize}
\item \verb|CameraController|: Permet moure la càmera i fer zoom.
\lstinputlisting[tabsize=2]{../Unity/Delta/Assets/Scripts/CameraController.cs}

\item \verb|dominoPosition|: Mou les peces de domino mentre les estas col·locant.
\lstinputlisting[tabsize=2]{../Unity/Delta/Assets/Scripts/dominoPosition.cs}

\item \verb|Draw|: Afegeix tota la funcionalitat de dibuixar circuits de domino.
\lstinputlisting[tabsize=2]{../Unity/Delta/Assets/Scripts/Draw.cs}

\item \verb|Editor|: Permet moure i rotar els objectes ja col·locats.
\lstinputlisting[tabsize=2]{../Unity/Delta/Assets/Scripts/Editor.cs}
\end{itemize}
Els scripts que s'han usat a \verb|SimuladorRobot| són:
\begin{itemize}
\item \verb|CameraController|: (el mateix que a Main).
\item \verb|RobotCreator|: Dibuixa el robot en 3D a partir dels paràmetres públics que se li passin.
\lstinputlisting[tabsize=2]{../Unity/Delta/Assets/Scripts/RobotCreator.cs}
\item \verb|Movements|: Modifica els paràmetres públics del robot per provar moviments determinats.
\lstinputlisting[tabsize=2]{../Unity/Delta/Assets/Scripts/Movements.cs}
\end{itemize}




